%\documentclass[%
%% border=1pt
%  border={0pt 0pt 0pt 0pt} % left bottom right top
%]{standalone}
%\usepackage{tikz} % Required for drawing custom shapes
%\usepackage{pgfplots}
%\pgfplotsset{compat = newest}
%
%\usepackage{amsbsy}
%\usetikzlibrary{decorations.pathreplacing,calc}
%\usetikzlibrary{shapes}
%\usepackage{amsmath,stackrel}
%\usetikzlibrary{arrows.meta}
%\usetikzlibrary{arrows}
%\usetikzlibrary{calc}
%\usetikzlibrary{math}
%
\begin{document}
\tikz{
%\begin{axis}[]
	\begin{axis}[enlarge x limits=0,
		enlarge y limits=0.13,
			xlabel={$t$},
		ylabel={$\Re \left\{x(t)\right\}$},
			width = 0.7\textwidth,
		height = 0.3\textwidth, ]
	\addplot[trig format =rad,black,domain=0:3,samples=200,thick]{cos(2*pi*x)}; % plot of real part of figure1.4
	%\addplot[black,domain=0:3,samples=200,thick]{sin(2*180*x)}; % plot of imaginary part of figure 1.4
	%\addplot[black,domain=0:10,samples=200,thick]{sqrt((cos(2*pi*x))^2+(sin(2*pi*x))^2)}; % plot of magnitude of figure 1.4
		%\addplot[black,domain=0:5,samples=2000,thick]{atan((tan(2*pi*x))}; % plot of angle of x of figure 1.4
	%\addplot[black,domain=0:10,samples=200,thick]{exp{-x}};
\end{axis}

	\begin{axis}[enlarge x limits=0,
		enlarge y limits=0.13,xshift=9cm,
			xlabel={$t$},
		ylabel={$\Im \left\{x(t)\right\}$},width = 0.7\textwidth,
		height = 0.3\textwidth,]
	\addplot[trig format =rad,black,domain=0:3,samples=200,thick]{sin(2*pi*x)}; % plot of imaginary part of figure 1.4

\end{axis}


\begin{axis}[enlargelimits=false,
	xshift=9cm,yshift=-3.5cm,ymin=0, ymax=1.5,
		xlabel={$t$},
	ylabel={$|x(t)|$},
	width = 0.7\textwidth,
	height = 0.3\textwidth,]
\addplot[trig format =rad,black,domain=0:5,samples=200,thick]{sqrt((cos(2*pi*x))^2+(sin(2*pi*x))^2)}; % plot of magnitude of figure 1.4

\end{axis}

\begin{axis}[enlarge x limits=0,
	enlarge y limits=0.13,xshift=9cm,yshift=-7cm,
	xlabel={$t$},
	ylabel={$\angle x(t)$},
	width = 0.7\textwidth,
	height = 0.3\textwidth,
	]
	%\addplot[trig format =rad,black,domain=0:5,samples=2000,thick]{atan((tan(2*pi*x))}; % plot of angle of x of figure 1.4
	\addplot[black,thick]
    table[col sep=comma]{Data_0_4.csv}; % plot of angle of x of figure 1.4
	
\end{axis}

\def\circ#1#2{
	\begin{scope}[shift={#1}, rotate=#2,scale=1.5,every node/.append style={transform shape}]
		\node[thick,circle,draw=black,minimum size=3cm] (A) at (0,0) {};
		\draw[thick,-{angle 60},black] (-2,0)--(2,0);
		\draw[thick,-{angle 60},black] (0,-2)--(0,2);
		\node[thick,below,inner sep=0.03cm,xshift=0.2cm] at (A.0) {$0$};
		\draw[thick,-{angle 60},black,rotate around={40:(0,0)}] (0,0)-- node[above,xshift=1cm]{} (1.5,0);
		\node[left,inner sep=0.03cm,xshift=-0.1cm,yshift=0.2cm] at (A.90) {$\frac{\pi}{2}$};
		\node[left,inner sep=0.03cm,xshift=-0.1cm,yshift=-0.2cm] at (A.270) {$-\frac{\pi}{2}$};
		\node[below,inner sep=0.03cm,xshift=-0.3cm,yshift=-0cm] at (A.180) {$-\pi$};
		\node[above,inner sep=0.03cm,xshift=-0.3cm,yshift=-0cm] at (A.180) {$\pi$};

	\end{scope}
}

\circ{(3,-4.3)}{0}

}
%\end{document}
