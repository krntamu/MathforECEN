\chapter{Geometric Series  - (
\href{https://youtu.be/PorDnSVfcqc}{Video},
\href{https://colab.research.google.com/drive/1iO7-QiDYePlRxaFbWY-plFB2ibVZd8rP?usp=sharing}{Python notebook})}

A sequence of the form $a, a r, \ldots, a r^{n},\ldots $, where $a$ and $r$ can be
 any {\em complex} number is called a geometric sequence.
 The first term in the sequence is $a$  and the ratio of any two adjacent terms is $r$, which is called the common ratio.
 The partial sum $S_N$ defined as
\[
S_N := a + a r + a r^2 + \ldots + a r^{N-1} = \sum_{n=0}^{N-1} a r^{n}
\]
is the sum of the first $N$ terms of the sequence and is called a geometric series.
Notice that the sum starts at $ar^0$ and goes up to $a r^{N-1}$.
Notice that lowercase $n$ is used as an index for the summation and uppercase $N$ is the number of terms.
It might seem annoying as to why I used lower case $n$ and upper case $N$ to mean two different things.
Throughout the course, we will have to sum signals which are indexed by time and it is common to use $n$ to represent a time index
and $N$ is commonly used to refer to the number of terms.
Therefore, it is better to get used to this notation.

A nice formula for $S_N$ can be obtained as follows
\begin{eqnarray}
S_N  & = & a + a r + a r^2 + \ldots + a r^{N-1}  \\
r S_N & = & \ \ \ \ \ \ a r + a r^2 + \ldots + a r^{N-1} + a r^N \\
(1-r)S_N & = & a - a r^N = a (1-r^N)
\end{eqnarray}
From this, we can see that
\begin{equation}
\label{eqn:sumtontermsgm}
\boxed{
S_N = \sum_{n=0}^{N-1} a r^n = \left\{
                       \begin{array}{ll}
                        \frac{a(1-r^{N})}{1-r}, & \hbox{$r \neq 1$;} \\
                        a N, & \hbox{$r = 1$.}
                       \end{array}
                     \right.\\
}
\end{equation}

Using the same idea as before, the following general formula can be derived
\begin{equation}
\label{eqn:geomsumgeneral}
\boxed{
		\hbox{For} \ N_2 > N_1, \sum_{n=N_1}^{N_2} r^n  = \left\{
                       \begin{array}{ll}
                        \frac{r^{N_1}-r^{N_2+1}}{1-r}, & {r \neq 1;} \\
                        N_2-N_1+1, & {r = 1.}
                       \end{array}
                     \right.\\
}
\end{equation}
The above formula is valid for any $N_2 > N_1$ regardless of whether $N_1,N_2$ are positive, zero or negative. While the formula for the sum a geometric sequence is straight forward, students often have difficulty recollecting this from memory. You {\em must memorize} this formula~$!$

The following special cases of (\ref{eqn:geomsumgeneral}) are typically encountered
\begin{equation}
\label{eqn:geomsp1}
\boxed{
		\sum_{n=0}^{N} r^n  = \left\{
                       \begin{array}{ll}
                        \frac{1-r^{N+1}}{1-r}, & {r \neq 1;} \\
                        N+1, & {r = 1.}
                       \end{array}
                     \right.\\
}
\end{equation}

Another special case of the above result is when $N_2 \rightarrow \infty$. In this case, the sum of infinite terms converges or diverges depending on whether $|r| < 1$ or $|r|> 1$.

\begin{equation}
\label{eqn:geomsp2}
\boxed{
\sum_{n=N_1}^{\infty} r^n = \left\{
                       \begin{array}{ll}
                        \frac{r^{N_1}}{1-r}, & \hbox{$|r| < 1$;} \\
                        \hbox{does not converge}, & \hbox{$|r| \geq 1$.}
                       \end{array}
                     \right.\\
                     }
\end{equation}

\begin{example}
For example, $1,\frac{1}{2},\frac{1}{4},\ldots$ is an infinite geometric series with $a=1$, $b=\frac12$.
\[
S_{11} = 1 + \frac{1}{2}+ \frac{1}{4} + \ldots + \ldots + \frac{1}{1024} = \frac{1-\frac{1}{2^{11}}}{1-\frac12}
\]
\end{example}

You may have seen these before, but in this class often we will be interested in the case when $a$ and/or $b$ are complex numbers. Luckily, nothing changes from when $a$ and $b$ are just real numbers.


An additional difficulty students often face is in recognizing that a given sum is actually a sum of a geometric sequence and in properly identifying $a$ and $r$. The practice problems in this section as well as the homework problems should give you some practice.

The following identity is also true, although we will not use this often in this class.
\begin{equation}
\label{sumgm2}
\sum_{n=0}^\infty n r^n = \frac{r}{(1-r)^2}, \ \ \ |r| < 1
\end{equation}

\section{Practice Problems -
(\href{http://signalsandsystems.wikidot.com/problems-geometric-series}{Video solutions})}

\begin{enumerate}

\item Compute $5+\frac{10}{3}+\frac{20}{9}+\frac{40}{27}+ \ldots$

\item Compute $5-\frac{10}{3}+\frac{20}{9}-\frac{40}{27}+ \ldots$

\item Simplify $\sum\limits_{n=2}^{9} 2^{3n} 3^{-2n}$

\item Compute $\sum\limits_{n=2}^{\infty} 2^{3n} 3^{-2n}$

\item Compute $\sum\limits_{n=-\infty}^{-2} 2^{-3n} 3^{2n}$. See if you can substitute $m = -n$ and obtain the expression in the previous problem

\item Simply $\sum\limits_{n=2}^{\infty} x^n 3^{-n}$ an expression as a rational function of $x$. Evaluate this function for $x=2$

\item Compute $\sum\limits_{n=1}^{\infty} \cos^n (\pi t)$ and express as a function of $t$

\item Compute $\sum\limits_{n=1}^{\infty} \frac{1}{2^n}\cos(n \pi t)$ and express the result as a function of $t$. Hint: Use Euler's formula to convert this sum of two geometric series.

\item Simplify $\sum\limits_{n=1}^{\infty} \left(\frac{1}{3} \right)^n e^{j \omega n}$

\item Compute $e^{j\frac{\pi}{2}}+\frac{1}{2} e^{{j \pi}} + \frac{1}{4} e^{j\frac{3 \pi}{2}} + \ldots + \frac{1}{2^9} e^{j\frac{10 \pi}{2}}$. Simplify your answer into a complex number in Cartesian form



\item Prove the result in (\ref{eqn:geomsp1}) and (\ref{eqn:geomsp2})
\item For any two given integers $k$ and $N$, what is
$\displaystyle{\sum_{n=0}^{N-1} e^{\frac{j 2 \pi k n}{N}}}$?

\item Just for intellectual curiosity - Can you prove the results in (\ref{eqn:sumtontermsgm}) and (\ref{sumgm2})?

\end{enumerate}
%----------------------------------------------------------------------------------------------------------------------------------------------------------------------

