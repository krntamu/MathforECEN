%\usepackage{subfiles}
\usepackage{makeidx}         % allows index generation
\usepackage{graphicx,wrapfig}        % standard LaTeX graphics tool
                             % when including figure files
\usepackage{multicol}        % used for the two-column index
\usepackage[bottom]{footmisc}% places footnotes at page bottom

%\usepackage{dsptricks}
%\usepackage{dspfunctions}
%\usepackage{dspblocks}

\usepackage{tikz}
\usepackage{pgfplots}
\usetikzlibrary{calc, arrows, shadows,shapes, arrows.meta, positioning, fit, decorations, intersections}
\tikzset{sig/.style={circle, draw, fill = white, inner sep = 0, minimum size = 3pt, node contents = {}}}
\usepackage{amsmath,amssymb,amsthm}
\usepackage{float}
\usepackage[a4paper, total={6in, 10in},left=25mm,top=20mm,]{geometry}
\usepackage{verbatim}
\usepackage{pdfpages}
\usepackage{grffile}
\usepackage{auto-pst-pdf}
\usepackage{pst-plot}
\usepackage[europeanresistors,europeaninductor]{circuitikz}
\usepackage{mdframed}
\usepackage{xcolor}
\definecolor{grey}{RGB}{0.5,0.5,0.5}
\usepackage{tcolorbox}
\usepackage{wrapfig}
\usepackage{fontawesome}
\usepackage{hyperref}
\hypersetup{
    colorlinks=true,
    linkcolor=blue,
    urlcolor=red,
    }
%\usepackage[Glenn]{fncychap}
\let\proof\relax
\let\endproof\relax

\newtcolorbox{warningbox}[2][]
{
  colframe = red!25,
  colback  = red!10,
  coltitle = red!20!black,
  title    = #2,
  #1,
}

% Hint tcolorbox
% #1: tcolorbox options
% #2: Box title
\newtcolorbox{hintbox}[2][]
{
  colframe = green!25,
  colback  = green!10,
  coltitle = green!20!black,
  title    = #2,
  #1,
}

% Info (information) tcolorbox
% #1: tcolorbox options
% #2: Box title
\newtcolorbox{infobox}[2][]
{
  colframe = blue!25,
  colback  = blue!10,
  coltitle = blue!20!black,
  title    = #2,
  #1,
}

%\usepackage{graphics,epsfig}
%\usepackage{psfrag}
\usepackage{polynom}
\def\pgfsysdriver{pgfsys-tex4ht.def}
\pgfplotsset{compat = newest}

\newtheorem*{remark}{Remark}
\newtheorem{theorem}{Theorem}[section]
\newtheorem{proposition}[theorem]{Proposition}
\newtheorem{definition}[theorem]{Definition}
\newtheorem{example}[theorem]{Example}
%\begin{comment}
\newcommand{\SetIn}{\ensuremath{\mathrm{1\!1}}}
\newcommand{\Expect}{\ensuremath{\mathrm{E}}}
\newcommand{\Var}{\ensuremath{\mathrm{Var}}}

\newcommand{\RealNumbers}{\mathbb{R}}
\newcommand{\RationalNumbers}{\mathbb{Q}}
\newcommand{\Integers}{\mathbb{Z}}
\newcommand{\NaturalNumbers}{\mathbb{N}}
\newcommand{\sinc}{\text{sinc}}
\newcommand{\tri}{\text{tri}}
\newcommand{\rect}{\text{rect}}

\newif\iflecture
\lecturefalse

%% Definitions
%%
\renewcommand{\baselinestretch}{1.25}
\newcommand{\defn}[2]{\textbf{\textrm{#2}}\index{#1!#2}}
%\end{comment}

\makeindex     % used for the subject index
                       % please use the style svind.ist with
                       % your makeindex program

